% !TEX TS-program = arara
% arara: xelatex: { synctex: on, options: [-halt-on-error] } 
%% arara: biber
%% arara: makeglossaries
%% arara: makeindex
%% arara: xelatex: { synctex: on, options: [-interaction=batchmode, -halt-on-error] } 
%% arara: xelatex: { synctex: on, options: [-interaction=batchmode, -halt-on-error]  } 
%% arara: clean: { extensions: [ aux, log, out, run.xml, toc, mw, synctex.gz, ] }
%% arara: clean: { extensions: [ bbl, bcf, blg, ] }
%% arara: clean: { extensions: [ glg, glo, gls, ] }
%% arara: clean: { extensions: [ idx, ilg, ind, xdy, ] }
%% arara: clean: { extensions: [ plCode, plData, plMath, plExercise, plNote, plQuote, ] }
%-----------------------------------------------------------------
\documentclass{PalisadesLakesArticle}
\geomLetter
\geomPortraitOneColumn
%-----------------------------------------------------------------
\title{Font tests}
\author{John Alan McDonald 
(palisades dot lakes at gmail dot com)}
\date{draft of \today}
%-----------------------------------------------------------------
\begin{document}
%\maketitle
%\PalisadesLakesTableOfContents
%-----------------------------------------------------------------
\def\sharedFolder{../shared/}

\begin{plSection}{Fonts}

Parindent should be zero:\\
\textsc{
The (quick) brown fox jumps over the lazy dog [0123456789].\\
THE (QUICK) BROWN FOX JUMPS OVER THE LAZY DOG [0123456789].\\
}
\textrm{
The (quick) brown fox jumps over the lazy dog [0123456789].\\
THE (QUICK) BROWN FOX JUMPS OVER THE LAZY DOG [0123456789].\\
}
\textlb{
The (quick) brown fox jumps over the lazy dog [0123456789].\\
THE (QUICK) BROWN FOX JUMPS OVER THE LAZY DOG [0123456789].\\
}
\textsb{
The (quick) brown fox jumps over the lazy dog [0123456789].\\
THE (QUICK) BROWN FOX JUMPS OVER THE LAZY DOG [0123456789].\\
}
\textbf{
The (quick) brown fox jumps over the lazy dog [0123456789].\\
THE (QUICK) BROWN FOX JUMPS OVER THE LAZY DOG [0123456789].\\
}
\textit{
The (quick) brown fox jumps over the lazy dog [0123456789].\\
THE (QUICK) BROWN FOX JUMPS OVER THE LAZY DOG [0123456789].\\
}
\textsl{
The (quick) brown fox jumps over the lazy dog [0123456789].\\
THE (QUICK) BROWN FOX JUMPS OVER THE LAZY DOG [0123456789].\\
}
\textsf{
The (quick) brown fox jumps over the lazy dog [0123456789].\\
THE (QUICK) BROWN FOX JUMPS OVER THE LAZY DOG [0123456789].\\
}
\texttt{
The (quick) brown fox jumps over the lazy dog [0123456789].\\
THE (QUICK) BROWN FOX JUMPS OVER THE LAZY DOG [0123456789].\\
}

$
\frac
{\partial f}{\partial x} 
\in \mathcal{F}\left(\mathbb{R}\right)$
;
$\nabla_{x} \mathsf{f} (y) \in \mathfrak{V}$

See 
\citeAuthorYearTitle[chapter 2]{Spivak:1965:CalculusOnManifolds}.


\begin{eqnarray}
\Derivative[\Vector{v}]{\Vector{f}}[\Vector{u}][\Vector{t}]
& = &
\Derivative[\Vector{v}_i]{\Vector{f}}
[(\Vector{u}_0 \ldots \Vector{u}_{n-1})][\Vector{t}_i] 
\\
\Partial{j}{\Vector{f}} & = &\Partial{v_j}{\Vector{f}} 
\\
\Derivative{\Vector{f}}[\Vector{u}]
& = &
\sum_{j=0}^{m-1} \Partial{j}{\Vector{f}}[\Vector{u}] 
\otimes \Vector{e}_j^{\Space{V}}
\\
\Partial{j}{\Vector{f}}[\Vector{u}]
& = &
\Derivative{\Vector{f}}[\Vector{u}] \Vector{e}_j^{\Space{V}}
\\
\Gradient{f}[\Vector{u}] 
& = &
\sum_{j=0}^{m-1} 
\left( 
\Partial{j}{f}[\Vector{u}] 
\right) \Vector{e}_j^{\Space{V}}
\\
\Derivative{\Vector{f}}[\Vector{u}]
& = &
 \sum_{i=0}^{n-1}  
\Vector{e}_i^{\Space{W}} 
\otimes \Gradient{\Vector{f}_i}[\Vector{u}]
\end{eqnarray}
\end{plSection}%{Fonts}
%\import{\sharedFolder}{spaces}
%\import{\sharedFolder}{derivatives}
%\import{\sharedFolder}{category}
%-----------------------------------------------------------------
\end{document}
%-----------------------------------------------------------------
