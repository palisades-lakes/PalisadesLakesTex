% !TEX TS-program = arara
% arara: xelatex: { synctex: on, options: [-halt-on-error] } 
%% arara: biber
%% arara: makeglossaries
%% arara: makeindex
%% arara: xelatex: { synctex: on, options: [-interaction=batchmode, -halt-on-error] } 
%% arara: xelatex: { synctex: on, options: [-interaction=batchmode, -halt-on-error]  } 
% arara: clean: { extensions: [ aux, log, out, run.xml, toc, mw, synctex.gz, ] }
%% arara: clean: { extensions: [ bbl, bcf, blg, ] }
%% arara: clean: { extensions: [ glg, glo, gls, ] }
%% arara: clean: { extensions: [ idx, ilg, ind, xdy, ] }
%% arara: clean: { extensions: [ plCode, plData, plMath, plExercise, plNote, plQuote, ] }
%----------------------------------------------------------------
\documentclass{PalisadesLakesArticle}

%\geomLegal
%\geomLedger
\geomHDTV
\geomLandscape
%-----------------------------------------------------------------
\begin{document}
%-----------------------------------------------------------------

\begin{plSection}{Tests}

\begin{plSection}{Colors}

\begin{plPlot}{}{}{}
\end{plPlot}
\begin{plTable}{}{}{}
\end{plTable}
\begin{plQuote}{}{}{}
\end{plQuote}
\begin{plAlgorithm}{}{}{}
\end{plAlgorithm}
\begin{plListing}{}{}{}
\end{plListing}

\begin{plNote}{}{}{}
\end{plNote}
\begin{plExercise}{}{}{}
\end{plExercise}

\begin{plDiagram}{}{}{}
\end{plDiagram}
\begin{plDefinition}{}{}{}
\end{plDefinition}
\begin{plExample}{}{}{}
\end{plExample}
\begin{plAxiom}{}{}{}
\end{plAxiom}
\begin{plAxiomSchema}{}{}{}
\end{plAxiomSchema}
\begin{plLemma}{}{}{}
\end{plLemma}
\begin{plTheorem}{}{}{}
\end{plTheorem}
\begin{plCorollary}{}{}{}
\end{plCorollary}

\end{plSection}%{Colors}

\begin{plSection}{The man called X}
\begin{plQuote}
{\citeAuthorYearTitle{Thurston:1994:Proof}}%
{thurston0}%
{Mathematics as we practice it is much more formally
complete and precise than
other sciences, but it is much less formally complete and precise
for its content than computer programs.
}
\end{plQuote}

\begin{plQuote}
{\citeAuthorYearTitle{Thurston:1994:Proof}}
{thurston1}
{On the most fundamental level, the foundations of mathematics 
are much shakier than the mathematics that we do. 
Most mathematicians adhere to foundational principles 
that are known to be polite fictions. 
For example, it is a theorem that there does not exist any way 
to ever actually construct or even define a well-ordering
of the real numbers. 
There is considerable evidence (but no proof) that we can get
away with these polite fictions without being caught out, 
but that doesn’t make them right. 
Set theorists construct many alternate and mutually contradictory
“mathematical universes” such that if one is consistent, 
the others are too. 
This leaves very little confidence that one or the other is 
the right choice or the natural choice. 
G\"odel’s incompleteness theorem implies 
that there can be no formal system that is consistent, 
yet powerful enough to serve as a basis 
for all of the mathematics that we do.}%
\end{plQuote}

See \cref{quote:thurston0} and \Cref{quote:thurston1}.

\end{plSection}%{Thurston}

\begin{plTable}{Random Table}{randomTable}
  \begin{tabular}{r|c|l}
  One   & Two    & Three \\\hline\hline
  Men   & Mice   & Lions \\\hline
  Upper & Middle & Lower
  \end{tabular}
  \end{plTable}

%-----------------------------------------------------------------
\begin{plSection}{Axiom of power set}
\label{sec:Axiom-of-power-set}

\begin{plDefinition}{Subset}{Subset}
The set $\Set{A}$ is a \textsl{subset} of $\Set{B}$:
$
(\Set{A} \subseteq \Set{B}) \;
 \Leftrightarrow \;
 (\forall q (q \in \Set{A} \, \Rightarrow \, q \in \Set{B})).
$.
\end{plDefinition}
 
\begin{plDefinition}{Power Set}{PowerSet}
The \emph{power set}, $\PowerSet{\Set{S}}$,
 is the set of all subsets of $\Set{S}$.
\end{plDefinition}

\begin{plAxiom}{Power Set Existence}{PowerSetExistence}
The \textsl{power set} axiom says the the power set
always exists.~\cite{wiki:AxiomOfPowerSet}
\end{plAxiom}

\begin{plNote}{}{PowerSetAxiom}
\Cref{axiom:PowerSetExistence} is controversial for infinite sets. 
Leads to paradoxes concerning sets that are ``too big'',
based on the idea that 
\begin{equation}
 \mathsf{cardinality}\left(\PowerSet{\Set{S}}\right)
 = 2^{\text{cardinality}(\Set{S})}
\end{equation}
Alternatively, one might think that $\mathsf{cardinality}$ of
infinite sets is the problem.
\end{plNote}

\end{plSection}%{Axiom of power set}
%-----------------------------------------------------------------
\begin{plSection}{Axiom schema of replacement}
\label{sec:Axiom-schema-of-replacement}

\begin{plAxiomSchema}{Replacement, version 1}{replacement1}
The image of a definable (?) function applied to a set
is contained in some set.\cite{wiki:AxiomSchemaOfReplacement}
(``image of set under function'' is usually defined as a set.)
aka axiom schema of collection.
\end{plAxiomSchema}%{Replacement, version 1}{replacement1}

\begin{plAxiomSchema}{Replacement, version 2}{replacement2}
The image of any set under a definable mapping is a set.

Suppose $P$ is a definable (?) relation such that for every
set $X$ there is a unique set $Y$ such that $P(X,Y)$.
$P$ may be a proper class\cite{wiki:ClassSetTheory}, 
ie, not a \textit{set}, of ordered pairs. 
(Leads to some hackiness about relations that are not sets,
for unconvincing reasons\cite{wiki:BinaryRelation}.)
Definable (class) function: $F_P(X)\,=\,Y \iff P(X,Y)$.
Collection $\Set{B}$ (may be a proper class)
defined so that for all $y\in\Set{B}$ there exist $x \in \Set{A}$
such that $y=F_P(x)$.
(Something backwards about this.)

$\Set{B}\,=\,F_p(\Set{A}) \,=\, \{F_p(x) : x \in \Set{A}\}$ 
is the \textsl{image} of $\Set{A}$ under $F_p$.

Principle of smallness: if $\Set{A}$ is 
``small enough'' to be a set,
then so is $F(\Set{A})$.
\end{plAxiomSchema}%{Replacement, version 2}{replacement2}

\Cref{axiomSchema:replacement1} 
and \cref{axiomSchema:replacement2} implied by stronger
axiom of limitation of 
size\cite{wiki:AxiomOfLimitationOfSize}:
a class that is a member of a class is a set (???!!!).

\nameref{sec:Axiom-schema-of-replacement} not needed for most math;
not present in \textsf{Z};
``drastically increases the strength of \textsf{ZF}''.

Used in proving theorems about ordinals.

Axiom schema of collection: some superclass of 
the image of relation is a set.
Stronger than replacement
in some axiom systems, weaker in others.
See also axiom schema of boundedness.

With axiom of empty set, replacement implies specification.,
using law of excluded middle.

Replacement is main thing distinguishing \textsf{Z}
from \textsf{ZF}.

Skoelem's 1st order version vs Fraenkel's $2$nd order?

\end{plSection}%{Axiom schema of replacement}
%-----------------------------------------------------------------
\begin{plSection}{Group}

\begin{plDefinition}{Group}{group}
Monoid $\left[ \Set{S}, \diamond \right]$ such that
\begin{description}
 \item[Inverse] For every $a \in \Set{S}$ there exists an
 $a^{-1}$ such that $a^{-1} \diamond a = i$.
\end{description}
\end{plDefinition}

\begin{plExercise}{}{Left inverse implies right inverse.}
Show that this implies that $a \diamond a^{-1} = i$ .
\end{plExercise} 

\end{plSection}%{Group}
%-----------------------------------------------------------------
\begin{plSection}{Generating spaces by 'closure'}

Generating these 'spaces' by closure under arithmetic operations
\cite{PickertGorke:1974:RealNumbers}.

A common pattern: 
a problem defined in terms of an existing mathematical
structure. 
Some combination of convenience, simplicity, computational
efficiency, and/or simplicity leads us to define a new structure, 
often a larger enclosing one.

We can use the number 'spaces' as an example.
Imagine, for the moment, that you only know the 
the natural numbers, $\Space{N}$, 
with a single operation: $+$.
Suppose we have a problem to solve:
\begin{math}
a + b = c
\end{math}
where $a,b,c\in \Space{N}$ 
and we know $b$ and $c$,
but not $a$.

\begin{plListing}
[breakable=false]
{Reinventing subtraction}
{reinventSubtraction}
\begin{lstlisting}
(when (<= $b$ $c$)
  (loop [$a$ 0]
    (cond 
      (= $c$ (+ $a$ $b$)) $a$
      (< $c$ (+ $a$ $b$)) :no-solution
      :else (recur (+ 1 $a$)))))
\end{lstlisting}
\end{plListing}

\end{plSection}%{Generating spaces by 'closure'}
%-----------------------------------------------------------------
\begin{plSection}{Linear spaces}
\label{sec:Linear-spaces}

\begin{plQuote}
{\citeAuthorYearTitle{MacLane:1954:Courses}}
{maclane:vspace}
{Throughout these courses the infusion of a geometrical
point of view is of paramount importance. A vector
is geometrical; it is an element of a vector space, defined
by suitable axioms—--whether the scalars be real numbers or
elements of a general field. A vector is not an n-tuple of
numbers until a coordinate system has been chosen. Any
teacher and any text book which starts with the idea that vectors
are n-tuples is committing a crime for which the proper
punishment is ridicule. The n-tuple idea is not ‘easier,’ it is
harder; it is not clearer, it is more misleading. By the same
token, linear transformations are basic and matrices are their
representations\ldots}
\end{plQuote}

My approach to linear (aka vector) spaces is largely based on
the texts I used as a college freshman for linear algebra and
multivariate calculus: \citeAuthorYearTitle{Halmos:1958:Finite}
and \citeAuthorYearTitle{Spivak:1965:CalculusOnManifolds}.

\begin{plDefinition}{Linear space}{linearSpace}
A \textit{linear space} 
$\Space{V} = 
\left[ \Set{V}, \Space{K}, \linearCombination \right]$
 is:
\begin{itemize}
  \item a set of \textit{vectors} $\Set{V}$,
  \item a field  of scalars $\Space{K}$,
  \item a linear combination function: 
\begin{equation}
\left( \linearCombination 
\, a_0 \, \Vector{v}_0 \, a_1 \, \Vector{v}_1 \right) \; 
 \rightarrow \; \Vector{v}_2  \in \Set{V}
\end{equation}
for $\Vector{v}_0, \Vector{v}_1 \in \Set{V} $
and $a_0, a_1 \in \Space{K}$.
Linear combination is often defined in terms of
$2$ binary operations:
scalar multiplication $a * \Vector{v} \in \Set{V}$,
and vector addition $\Vector{v}_0 + \Vector{v}_1 \in \Set{V}$:
\begin{equation}
\left( \linearCombination 
\, a_0 \, \Vector{v}_0 \, a_1 \, \Vector{v}_1 \right) \; 
= \; a_0*\Vector{v}_0 + a_1*\Vector{v}_1
\end{equation}
\textbf{TODO:} required identities for $+$ and $*$ 
from Spivak or Halmos.
\end{itemize}
\end{plDefinition}

Usually the distinction between $\Set{V}$ and $\Space{V}$ 
is ignored, and we will say, for example, 
$\Vector{v} \in \Space{V}$.

\begin{plDefinition}{Linear dependence}{}
Suppose $\Space{V}$ is a linear space and
$\Vector{v}_0, \ldots , \Vector{v}_{n-1} \in \Space{V}$.
If there exists $a_0, \ldots , a_{n-1}$ such that
$\Vector{0} = \sum a_i \Vector{v}_i$ then the $\{ \Vector{v}_i \}$
are \textit{linearly dependent}.
\cite[section~5]{Halmos:1958:Finite}

Otherwise they are \textit{linearly independent}.
\end{plDefinition}

\begin{plLemma}{The set of non-zero vectors
 $\Vector{v}_0, \ldots ,\Vector{v}_{n-1}$
is linearly dependent iff some $\Vector{v}_k, \; 1 \leq n-1$, 
is a linear combination of the precceeding 
ones~\cite[Section 6]{Halmos:1958:Finite}.}{linearDependenceLemma}
\textsc{Proof:}
Assume  $\Vector{v}_0, \ldots ,\Vector{v}_{n-1}$ are linearly dependent.
Consider the smallest $k$ such that 
$\Vector{v}_0, \ldots ,\Vector{v}_{k}$ is linearly dependent.
By definition,
there exists non-zero $a_0, \ldots ,a_{k}$ such that
\begin{equation}
0 = \sum_0^k a_i \Vector{v}_i
\end{equation}
which implies that
\begin{equation}
\Vector{v}_k = \sum_0^{k-1} \frac{a_i}{- a_k} \Vector{v}_i
\end{equation}
\end{plLemma}

\begin{plTheorem}{The set of non-zero vectors
 $\Vector{v}_0, \ldots ,\Vector{v}_{n-1}$
is linearly dependent iff some $\Vector{v}_k, \; 1 \leq n-1$, 
is a linear combination of the precceeding 
ones~\cite[Section 6]{Halmos:1958:Finite}.}{linearDependenceTheorem}
\textsc{Proof:}
Assume  $\Vector{v}_0, \ldots ,\Vector{v}_{n-1}$ are linearly dependent.
Consider the smallest $k$ such that 
$\Vector{v}_0, \ldots ,\Vector{v}_{k}$ is linearly dependent.
By definition,
there exists non-zero $a_0, \ldots ,a_{k}$ such that
\begin{equation}
0 = \sum_0^k a_i \Vector{v}_i
\end{equation}
which implies that
\begin{equation}
\Vector{v}_k = \sum_0^{k-1} \frac{a_i}{- a_k} \Vector{v}_i
\end{equation}
\end{plTheorem}

\begin{plCorollary}{The set of non-zero vectors
 $\Vector{v}_0, \ldots ,\Vector{v}_{n-1}$
is linearly dependent iff some $\Vector{v}_k, \; 1 \leq n-1$, 
is a linear combination of the precceeding 
ones~\cite[Section 6]{Halmos:1958:Finite}.}{linearDependenceCorollary}
\textsc{Proof:}
Assume  $\Vector{v}_0, \ldots ,\Vector{v}_{n-1}$ are linearly dependent.
Consider the smallest $k$ such that 
$\Vector{v}_0, \ldots ,\Vector{v}_{k}$ is linearly dependent.
By definition,
there exists non-zero $a_0, \ldots ,a_{k}$ such that
\begin{equation}
0 = \sum_0^k a_i \Vector{v}_i
\end{equation}
which implies that
\begin{equation}
\Vector{v}_k = \sum_0^{k-1} \frac{a_i}{- a_k} \Vector{v}_i
\end{equation}
\end{plCorollary}
%-----------------------------------------------------------------
\end{plSection}%{Linear spaces}
%-----------------------------------------------------------------
\begin{plQuote}
{Dedekind to Klein, April 1888\\ 
from\\ \citeAuthorYearTitle{Dugac:1976:DedekindFondements},\\
as quoted in\\ \citeAuthorYearTitle{Ferreiros:2007:Labyrinth}.}
{dedekindklein}
{\fraklines
\textgoth{Und was wird der geduldige Leser am
Schlusse sagen? Dass der Verfasser mit einem Aufwande von 
uns/iglicher Arbeit es gliicklich
erreicht hat, die klarsten Vorstellungen 
in ein unheimliches Dunkel zu hiillen!}
\par
\textswab{Und was wird der geduldige Leser am
Schlusse sagen? Dass der Verfasser mit einem Aufwande von 
uns/iglicher Arbeit es gliicklich
erreicht hat, die klarsten Vorstellungen 
in ein unheimliches Dunkel zu hiillen!}
\par
\textfrak{Und was wird der geduldige Leser am
Schlusse sagen? Dass der Verfasser mit einem Aufwande von 
uns/iglicher Arbeit es gliicklich
erreicht hat, die klarsten Vorstellungen 
in ein unheimliches Dunkel zu hiillen!}
\par
% \textinit{Und was wird der geduldige Leser am
% Schlusse sagen? Dass der Verfasser mit einem Aufwande von 
% uns/iglicher Arbeit es gliicklich
% erreicht hat, die klarsten Vorstellungen 
% in ein unheimliches Dunkel zu hiillen!}
% \par
}
What will the forbearing reader say at the end? That the author, 
in a squandering of indescribable
work, has happily managed to surround the clearest ideas
 in a disturbing obscurity!
\end{plQuote}

\begin{plDefinition*}{Directed graph}{digraph}
A \ding{directed (multi)graph} (aka \ding{digraph}) is
an ordered pair $G=\left( \Set{V}, \Set{E} \right)$,
where $\Set{V}$ is a set of \ding[vertex]{vertices}, 
$\Set{E}$ a set of \ding[edge]{edges}, ordered pairs of vertices.
$V_0 = \tail(e)$ and $V_1 = \head(e)$
for edge $e = \left[ V_0, V_1 \right]$.
\end{plDefinition*}

See \cref{definition:linearSpace} 
and \Cref{definition:linearSpace}.

%-----------------------------------------------------------------
\begin{plSection}{Category}

\begin{plDefinition}{Category}{category}
A \textsl{category} consists of: 
\begin{itemize}
\item A base set, $\Set{O}$, 
whose elements are ususally refered to as \textit{objects}.
\item A second set, $\Set{M}$, of \textit{morphisms}, 
each of which depends on an ordered pair 
$\left( \text{source} , \text{target} \right)$ of elements of 
$\Set{O}$. 
\item A \textsl{composition} operation, $\circ$, on 
the subset of the pairs of morphisms $f , g \in \Set{M}$
where $\text{source}(f) = \text{target}(g)$, satisfying:
\begin{description}
\item[Associativity:] $f \circ g \circ g \defeq 
\left( f \circ g \right) \circ h 
= f \circ \left( g \circ h \right)$
\item[Identity:] For each $x \in \Set{O}$, there exists an element
$1_x \in \Set{M}$ such that 
$x = \text{source}(1_x) = \text{target}(1_x)$, and 
$1_x \circ f = f$ (when $x = \text{target(f)}$) and
$f \circ 1_x = f$ (when $x = \text{source(f)}$).
\end{description}
\end{itemize}
\end{plDefinition}

 \begin{plExample}{The Set category}{}
 $\Set{O} = $ some collection of sets;\\
 $\Set{M} = $ the functions between those sets;\\
 $\circ = $ function composition.  
 \end{plExample}
 
\begin{plDiagram}[nofloat]
{A directed (acyclic) graph}
{digraph}
\centering
\begin{tikzcd}[
row sep=1.2cm, 
column sep=2.5cm,
ampersand replacement = \&,
%execute at begin picture={
%     \useasboundingbox (-5.15,-3.35) rectangle (4.25,3.4); }
    ]
V_0
\arrow[drr, "e_{02}", bend left]
\arrow[ddr, "e_{03}"', bend right]
\arrow[dr, "{e_{01}}"] 
\& 
\& 
\\
\& 
V_1 
\arrow[r, "e_{12}"] 
\arrow[d, "e_{13}"']
\& 
V_2 
\arrow[d, "e_{24}"] 
\\
\& 
V_3 
\arrow[r, "e_{34}"']
\& V_4
\end{tikzcd}
\end{plDiagram}

\Cref{diagram:aCompletedDigraph} shows the additional
edges needed to make the graph in \cref{diagram:digraph}
into a category: the identity loops are in blue, first order
compositions red, and second order green.

% Note tkzcd bug: bounding box for diagram includes the control 
% pts of the curved edges! 
\begin{plDiagram}
{Completion of \cref{diagram:digraph} to a category}
{aCompletedDigraph}
\centering
%\fbox{
\begin{tikzcd}[
row sep=1.2cm, 
column sep=2.5cm,
ampersand replacement = \&,
execute at begin picture={
     \useasboundingbox (-5.15,-3.35) rectangle (4.25,3.4); }
    ]
V_0
\arrow[loop left, blue, crossing over]
%\arrow[drr, bend left=40, "e_{02}" near start]
%\arrow[ddr, bend right=40, "e_{03}"' near start]
\arrow[drr, bend left, "e_{02}" near start]
\arrow[ddr, bend right, "e_{03}"' near start]
\arrow[dr, "{e_{01}}" ] 
\arrow[drr, "e_{12} \comop e_{01}", bend left=10, dashed, red]
\arrow[ddr, "e_{13} \comop e_{01}", bend right=10, dashed, red]
\& 
\& 
\\
\& 
V_1 
\arrow[loop above, blue,crossing over]
\arrow[r, "e_{12}"] 
\arrow[d, "e_{13}"']
\arrow[dr, bend left, "e_{24} \comop e_{12}", dashed, red]
\arrow[dr, bend right, "e_{34} \comop e_{13}"' , dashed, red]
\& 
V_2 
\arrow[loop right, crossing over, blue]
\arrow[d, "e_{24}"] 
\\
\& 
V_3 
\arrow[loop below, blue,crossing over]
\arrow[r, "e_{34}"']
\&
V_4
\arrow[loop right, blue,crossing over]
\arrow["e_{24} \comop e_{02}" {near end},
from=1-1,to=3-3, red, dashed, bend left=100,]
\arrow["e_{34} \comop e_{03}"' {near end},
from=1-1,to=3-3, red, dashed, bend right=100,]
\arrow["e_{24} \comop e_{12} \comop e_{01}" {near start},
from=1-1,to=3-3, draw=DarkGreen, dashed, bend left=120]
\arrow["e_{34} \comop e_{13} \comop e_{01}"' {near start},
from=1-1,to=3-3, draw=DarkGreen, dashed, bend right=120]
\end{tikzcd}
%}
\end{plDiagram}

\begin{plDiagram}
{Commuting version of \cref{diagram:aCompletedDigraph}}
{aCommutingDiagram}
\centering
\fbox{
\begin{tikzcd}
[
row sep=1.2cm, 
column sep=2.5cm,
ampersand replacement = \&,
execute at begin picture={
     \useasboundingbox (-3.3,-2.5) rectangle (4.25,3.4); }
    ]
V_0
\arrow[loop above, blue, crossing over]
\arrow[dr, "{e_{01}}" ] 
\arrow[drr, "e_{12} \comop e_{01}"', "e_{02}"  {near start},
bend left]
\arrow[ddr, "e_{13} \comop e_{01}", "e_{03}"' {near start},
bend right]
\& 
\& 
\\
\&
V_1 
\arrow[loop above, blue,crossing over]
\arrow[r, "e_{12}"] 
\arrow[d, "e_{13}"']
\arrow[dr, "e_{24} \comop e_{12}", "e_{34} \comop e_{13}"',
dashed, red]
\& 
V_2 
\arrow[loop right, crossing over, blue]
\arrow[d, "e_{24}"] 
\\
\& 
V_3 
\arrow[loop below, blue,crossing over]
\arrow[r, "e_{34}"']
\& 
V_4
\arrow[loop right, blue, crossing over]
\arrow[
"e_{24} \comop e_{02}",
"e_{34} \comop e_{03}" {near end},
"e_{24} \comop e_{12} \comop e_{01}" {near start},
"e_{34} \comop e_{13} \comop e_{01}"' {near start},
from=1-1,to=3-3, draw=DarkGreen, dashed, bend left=100,]
\end{tikzcd}
}
\end{plDiagram}
 
\end{plSection}%{Category}
%-----------------------------------------------------------------
\end{plSection}%{Tests}
%-----------------------------------------------------------------
\end{document}
%-----------------------------------------------------------------

