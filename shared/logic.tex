%-----------------------------------------------------------------
\begin{plSection}{Logic}
\label{sec:Logic}

\citeAuthorTitle{wiki:Logic}

\citeAuthorYearTitle{Ferreiros2001RoadModernLogic}

Axioms with quantifiers vs axiom schemas?

Relationship between $\exists, \forall$ quantifiers
and general iteration/reduction over sets?

Is there such a thing as no-order logic?
Self-referential definition of sets, relations, functions,
(and arithmetic) etc. 
Logical system  as just another mathematical structure:
sets plus functions plus constraints on function values.

Church and Turing vs G\"{o}del.

Tarski.

%-----------------------------------------------------------------
\begin{plSection}{Classical Logic}
\label{sec:Classical Logic}

\citeAuthorYearTitle{sep:ClassicalLogic}

\end{plSection}%{Classical Logic}
%-----------------------------------------------------------------
\begin{plSection}{Formal languages}
\label{sec:Formal_languages}

\citeAuthorYearTitle{DutilhNovaes:2012:FormalLanguages}

\end{plSection}%{Formal languages}
%-----------------------------------------------------------------
\begin{plSection}{Proof theory}
\label{sec:Proof_theory}

\end{plSection}%{Proof theory}
%-----------------------------------------------------------------
\begin{plSection}{Model theory}
\label{sec:Model_theory}

\end{plSection}%{Model theory}
%-----------------------------------------------------------------
\begin{plSection}{Zeroth order logic}
\label{sec:Zeroth_order_logic}

Classical zeroth-order (propositional) 
logic~\citeAuthorYearTitle{iep:PropositionalLogic,
wiki:PropositionalCalculus,
wiki:ZerothOrderLogic}

Subtle difference sometimes between propositional
and zeroth-order logic 
(=binary truth functional propositional logic).

Formal system:
language with atomic symbols
and logical operators defining well-formed formulas,
and inference rules that a set of axiom formulas 
and return deduced formulas.
\begin{description}
\item[Atoms]  $A, B, \ldots P, Q, \ldots$, 
that may be assigned values \textsf{true} or \textsf{false}.

\item[Operators] $\lnot P$, $P \wedge Q$, $P \vee Q$, 
$P \Rightarrow Q$, $P \Leftrightarrow Q$. 
Minimal set is $\lnot$ and 
any $2$ of  $\wedge, \vee, \Rightarrow$;
sufficient to define $2$ remaining operators.

\item[Propositions] (Well formed formulas)
Recursively: atoms and propositions combined with operators,
including parens for grouping: $(P \wedge Q) \Rightarrow R$.

\item[Inference rules] ${P \Rightarrow Q, P} \vdash Q$:
Takes a set of propositions (with truth assignments?)
and returns another proposition.
\end{description}

A variety of such formal languages.

Syntatic entailment: formula derived from set of axioms 
using inference rules in a finite number of steps.

Semantic entailment: formula evaluates to \textsf{true}
under all possible truth assignments to variables
in axioms.

Issues are 
\begin{description}
\item[Soundness]
All syntactically entailed formulas
are semantically entailed, that is,
a sequence of inferences will never lead to a contradiction.

\item[Completeness] 
all semantically entailed formulas 
are syntactically entailed, that is,
any statement which is consistent with the axioms under all 
possible truth assignments is derivable (somehow) using the 
inference rules.
\end{description}soundness 

%-----------------------------------------------------------------
\begin{plSection}{Intuitionist zeroth order logic}
\label{sec:Intuitionist_zeroth_order_logic}

\citeAuthorTitle{wiki:IntuitionisticLogic}

No law of excluded middle~\citeAuthorTitle{wiki:LawOfExcludedMiddle}.

\end{plSection}%{Intuitionist zeroth order logic}
%-----------------------------------------------------------------
\end{plSection}%{Zeroth order logic}
%-----------------------------------------------------------------
\begin{plSection}{Modal logic}
\label{sec:Modal_logic}

\citeAuthorTitle{wiki:ModalLogic}

Possibility vs necessity.

Epistemic: state of knowledge

Temporal 

\end{plSection}%{Modal logic}
%-----------------------------------------------------------------
\begin{plSection}{Paraconsistent logic}
\label{sec:Paraconsistent_logic}

\citeAuthorTitle{wiki:Paraconsistent_logic}

Get rid of "from a contradiction, anything follows".
\end{plSection}%{Paraconsistent logic}
%-----------------------------------------------------------------
\begin{plSection}{First order logic}
\label{sec:First_order_logic}

First-order (aka predicate) logic.~\citeAuthorTitle{wiki:FirstOrderLogic,
sep:EmergenceOfFirstOrderLOgic}

\begin{description}
\item[Truth values] Usually $\{\mathsf{true},\mathsf{false}\}$,
but more than $2$ values is possible.
\item[Domain of discourse] A set $\Set{D}$.
\item[Constants] $A,B,C, \ldots$ names for values in the
domain.
\item[Variables] $a,b,c, \ldots x,y,\dots$ taking on values in the
domain.~\citeAuthorTitle{wiki:FreeVariablesAndBoundVariables}
\item[Equality] combine atoms and variables.
\item[Functions] combine atoms and variables.
\item[Quantifiers]   
$\exists x \in \Set{S}$, $\forall x \in \Set{S}$ 
(where $\Set{S} \subseteq \Set{D}$);
$\exists x$ means $\exists x \in \Set{D}$, 
and
$\forall x$ means $\forall x \in \Set{D}$) 
bind variables in
formulas.~\citeAuthorTitle{wiki:QuantifierLogic}
More specialized quantifiers are possible:
$\exists \textrm{ a unique } x \in \Set{S}$
\item[Predicates] combine atoms and variables.
\end{description} 

Differentiate 1st order \textit{language} 
(no domain of discourse, just symbols)
from \textit{interpretation} 
(assignment of terms to elements of $\Set{D}$).

See also models/structures.~\citeAuthorTitle{wiki:ModelTheory}

(Circularity w.r.t set theory?
Zermelo-Fraenkel uses 1st order logic to define sets,
but 1st order logic is defined using sets\ldots?)

%-----------------------------------------------------------------
\begin{plSection}{First order language}
\label{sec:First_order_language}

Syntactic rules without domain of discourse.

Signature: arities of predicates and functions.

Prenex normal form (PNF):~\citeAuthorTitle{wiki:PrenexNormalForm} 
\textit{prefix} containing all quantifiers 
followed by quantifier-free \textit{matrix}.

In classical logic, every wff has a prenex equivalent;
not true for intuitionistic logic.

Example:
$\forall x 
((\exists y\phi (y))
\lor 
((\exists z\psi (z))\rightarrow \rho (x)))$ is not prenex.
$\forall x\exists y\forall z
(\phi (y)\lor (\psi (z)\rightarrow \rho (x)))$ 
is equivalent prenex. 
(does this fail in intuitionistic logic?)

\end{plSection}%{First order language}
%-----------------------------------------------------------------
\begin{plSection}{First order model}
\label{sec:First_order_model}

Domain of discourse and evaluation of formulas.

Equality.
(does this belong here? can also be axioms on a theory.)

\end{plSection}%{First order model}
%-----------------------------------------------------------------
\begin{plSection}{First order theory}
\label{sec:First_order_theory}

Axioms (and axiom schemas).~\citeAuthorTitle{wiki:ListOfFirstOrderTheories}

Consistency: no contradiction derivable from axioms.

Completeness: any formula can be proven 
\textsf{true} or \textsf{false} 
(axioms plus formula permit derivation of 
\textsf{true} or \textsf{false} but not both).

\end{plSection}%{First order theory}
%-----------------------------------------------------------------
\begin{plSection}{First order deductive systems}
\label{sec:First_order_deductive_systems}

Rules of inference.

\end{plSection}%{First order deductive systems}
%-----------------------------------------------------------------
\begin{plSection}{Monadic first order logic}
\label{sec:Monadic_first_order_logic}

All predicates and functions are unary 
(1 argument).~\citeAuthorTitle{wiki:MonadicPredicateCalculus}

Decidable, not very expressive.

\end{plSection}%{Monadic first order logic}
%-----------------------------------------------------------------
\begin{plSection}{Many sorted first order models}
\label{sec:Many_sorted_first_order_models}

Hilbert's function calculus.~\citeAuthorYearTitle{sep:EmergenceOfFirstOrderLOgic}

Domain of discourse has multiple sets;
variables 'typed' as having values in some particular 
set.~\citeAuthorYearTitle{sep:modeltheory_fo}

(Is this really different from
axiom restricting variable to subset of domain?()

\end{plSection}%{Many sorted first order models}
%-----------------------------------------------------------------
\end{plSection}%{First order logic}
%-----------------------------------------------------------------
\begin{plSection}{Second order logic}
\label{sec:Second_order_logic}

Second-order logic~\citeAuthorTitle{wiki:SecondOrderLogic,
wiki:SecondOrderPropositionalLogic}

\end{plSection}%{Second order logic}
%-----------------------------------------------------------------
\begin{plSection}{Higher order logic}
\label{sec:Higher_order_logic}

Higher-order logic~\citeAuthorTitle{wiki:HigherOrderLogic}

\end{plSection}%{Higher order logic}
%-----------------------------------------------------------------
\begin{plSection}{Constructivism}
\label{sec:Constructivism_logic}

\citeAuthorYearTitle{Feferman:2000:ConstructivePredicativeClassicalAnalysis,Diez:2002:Constructivism,sep:ConstructiveMathematics,
wiki:ConstructivismPhilosophyOfMathematics}

As opposed to classical logic~\citeAuthorTitle{wiki:ClassicalLogic}.
\end{plSection}%{Constructivism}
%-----------------------------------------------------------------
\end{plSection}%{Logic}
%-----------------------------------------------------------------
 