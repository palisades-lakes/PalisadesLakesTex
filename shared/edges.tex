\begin{plSection}{Functions of the edges}
\label{sec:edges}

\begin{plDiagram}
{Edge face pair labeling.}
{EdgeFaces}
\centering
\begin{verbatim}
          p1
          o
         /|\
        / | \
       /  |  \e31
   e12/   |   \
     /    |e01 \
    /     |     \
p2 o f012 | f031 o p3
    \     |     /
     \    |    /
   e20\   |   /e03
       \  |  /
        \ | /
         \|/
          o
          p0
\end{verbatim}
\end{plDiagram}

Notation in this section is based on \cref{diagram:EdgeFaces}.
We are discussing functions defined on a neighborhood of edge $e_{01}$.

We assume that, for each edge, an arbitrary order is assigned to
its two vertices, which are then at the positions $\Point{p}_0,\Point{p}_1$ in the diagram.

An interior edge has 2 adjacent faces, $f_{012}$ and $f_{031}$.
We assume that these 2 faces are oriented consistently, with the labels
taken counterclockwise, so that the normal vectors point out of the page.
Each face is represented by an ordered triple of vertices,
but the order is only determined up to a circular permutation;
for example, $f_{012}$ may be represented by the ordered triples
$(\Point{p}_0,\Point{p}_1,\Point{p}_2)$, 
$(\Point{p}_2,\Point{p}_0,\Point{p}_1)$, or 
$(\Point{p}_1,\Point{p}_2,\Point{p}_0)$,
but not by
$(\Point{p}_0,\Point{p}_2,\Point{p}_1)$, 
$(\Point{p}_1,\Point{p}_0,\Point{p}_2)$, or
 $(\Point{p}_2,\Point{p}_1,\Point{p}_0)$.

For the given ordering $(\Point{p}_0,\Point{p}_1)$ of the edge,
$f_{120}$ is the edge's {\it left face}
and $f_{031}$ is the {\it right face}.

Note that we cannot assume any consistent ordering of the 4 neighboring edges;
for example, $e_{12}$ may be represented by either ordered pair
$(\Point{p}_1,\Point{p}_2)$ or $(\Point{p}_2,\Point{p}_1)$.

%-----------------------------------------------------------------
\begin{plSection}{Edge length}
\label{sec:edge_length}

The edge tangent vector is $\Point{p}_1 - \Point{p}_0$.

The gradient of its squared length is:
\begin{equation}
\Gradient[\Point{p}_i]{\| \Point{p}_1 - \Point{p}_0 \|^2}[\Point{q}] 
= 2 \left( \Point{p}_i - \Point{p}_{(i+1) \bmod 1} \right)
\end{equation}

The gradient of the edge length, $\|\Point{p}_1 - \Point{p}_0\|$ is:
\begin{equation}
\Gradient[\Point{p}_i]{\| \Point{p}_1 - \Point{p}_0 \|}[\Point{q}] =
\frac{\left( \Point{p}_i - \Point{p}_{(i+1) \bmod 1} \right)}
{\|\Point{p}_1 - \Point{p}_0\|}
\end{equation}

\end{plSection}%{Edge length}
%-----------------------------------------------------------------

\begin{plSection}{Edge dihedral}
\label{sec:edge_dihedral}

In smooth surfaces,
the standard measures of curvature are all measures
of the rate of change of the surface normal vector.
For triangular meshes, it is therefore natural to
consider functions that depend on the change in
normal vectors between nearby faces.

In this section, we consider functions of the difference
in normals for 2 faces that share an edge,
which is equivalent to the {\it dihedral angle} of the edge.

We will consider a number of both real- and $\Reals^3$-valued functions on
$\Reals^{12} = \Reals^3 \oplus \Reals^3 \oplus \Reals^3 \oplus \Reals^3$.
We will use 
$\Point{p} = (\Point{p}_0, \Point{p}_1, \Point{p}_2, \Point{p}_3)$, 
$\Point{q}=\ldots$, $\Point{r}=\ldots$, etc.,
to refer to the arguments of these functions, with the meaning
of the indices determined by \cref{diagram:EdgeFaces}.
We are also interested in the two 9-dimensional subspaces
corresponding to the vertics of the two faces:
$\Point{p}_{012} = 
(\Point{p}_0,\Point{p}_1,\Point{p}_2), \Point{p}_{031} = 
(\Point{p}_0,\Point{p}_3,\Point{p}_1)$.
We abbreviate the two normal vectors:
$\Normal{n}_{012} = \Normal{n}(\Point{p}_0,\Point{p}_1,\Point{p}_2)$
and
$\Normal{n}_{031} = \Normal{n}(\Point{p}_0,\Point{p}_3,\Point{p}_1)$.

\end{plSection}%{Edge dihedral}
%-----------------------------------------------------------------

\begin{plSection}{Difference in face normals}
\label{sec:normal_difference}

One measure of the change in surface normal across an edge
is simply the vector difference of the two normals:

\begin{equation}
\label{eq:deltan}
\DNormal{n} (\Point{p}_0, \Point{p}_1, \Point{p}_2, \Point{p}_3)
=
\Normal{n} (\Point{p}_{012}) - \Normal{n} (\Point{p}_{031})
\end{equation}

The (total) derivative of the squared distance 
between adjacent face normals is:
\begin{eqnarray}
\label{eq:deltan_derivative}
\Derivative{\|\DNormal{n}(\Point{p})\|^2}[\Point{q}]
& =
2 \ \DNormal{n} ( \Point{q} )^\dagger &
\left( \Derivative{ ( \DNormal{n} ) }[\Point{q}] \right)
\\
& =
2 \ \DNormal{n}(\Point{q})^\dagger &
\left( 
\Derivative{\Normal{n}(\Point{p}_{012})}[\Point{q}] 
- \Derivative{\Normal{n}(\Point{p}_{031})}[\Point{q}]
 \right)
\nonumber \\
& =
2 \DNormal{n}(\Point{q})^\dagger &
\left\{ \; 
\left[ \Identity_{\Reals^3}
 - \left(
  \Normal{n}( \Point{q}_{012} ) 
  \otimes \Normal{n}( \Point{q}_{012} )
   \right)
\right]
\ast 
\Derivative{\Vector{a} \left( \Point{p}_{012} \right) }[\Point{q}]
\right.
\nonumber \\
&
& 
\left.
- \left[ \Identity_{\Reals^3} 
- \left( \Normal{n}( \Point{q}_{031} ) \otimes \Normal{n}
 ( \Point{q}_{031} ) \right)
\right]
\ast \Derivative{\Vector{a} ( \Point{p}_{031} ) }[\Point{q}]
\; \right\}
\nonumber
\end{eqnarray}

The partial derivatives, with respect to one of the vertices,
like 
$\Derivative[\Point{p}_0]{\Vector{a} (\Point{p}_{012})}
[\Point{q}][\Point{r}_0]$,
all have a similar form:
\begin{equation}
\Derivative[\Point{p}_0]{\Vector{a} (\Point{p}_{012})}
[\Point{q}][\Point{r}_0] 
 = (\Point{q}_1 - \Point{q}_3) \times \Point{r}_0
\end{equation}
Using this, \cref{eq:deltan_derivative}, \cref{eq:dot_cross},
and the facts that
$\DNormal{n}(\Point{q})  \perp  \Normal{n}(\Point{q}_{012}) = 
- \left( 
\Normal{n}(\Point{q}_{031})  
\perp  
\Normal{n}(\Point{q}_{012}) 
\right)$
and
$\DNormal{n}(\Point{q})  \perp  \Normal{n}(\Point{q}_{031}) =
 \Normal{n}(\Point{q}_{012})  \perp  \Normal{n}(\Point{q}_{031})$,
we can write the partial gradients without reference to the
derivative's argument $\Point{r}$:
\begin{eqnarray}
\label{eq:normal-difference-gradient}
\Gradient[\Point{p}_0]{\|\DNormal{n}\|^2}[\Point{q}]
& = &
\left[
{\frac{
\Normal{n}(\Point{q}_{031})  \perp  \Normal{n}(\Point{q}_{012})
}
{A(\Point{q}_{012})}
}
\times (\Point{q}_1 - \Point{q}_2)
\right]
\; + \;
\left[
{\frac{
\Normal{n}(\Point{q}_{012})  \perp  \Normal{n}(\Point{q}_{031})
}
{A(\Point{q}_{031})}
}
\times (\Point{q}_3 - \Point{q}_1)
\right]
\nonumber \\
\Gradient[\Point{p}_1]{\|\DNormal{n}\|^2}[\Point{q}]
& = &
\left[
{\frac{
\Normal{n}(\Point{q}_{031})  \perp  \Normal{n}(\Point{q}_{012})
}
{A(\Point{q}_{012})}
}
\times (\Point{q}_2 - \Point{q}_0)
\right]
\; + \;
\left[
{\frac{
\Normal{n}(\Point{q}_{012})  \perp  \Normal{n}(\Point{q}_{031})
}
{A(\Point{q}_{031})}
}
\times (\Point{q}_0 - \Point{q}_3)
\right]
\nonumber
\\
\Gradient[\Point{p}_2]{\|\DNormal{n}\|^2}[\Point{q}]
& = &
\left[
{\frac{
\Normal{n}(\Point{q}_{031})  \perp  \Normal{n}(\Point{q}_{012})
}
{A(\Point{q}_{012})}
}
\times (\Point{q}_0 - \Point{q}_1)
\right]
\nonumber
\\
\Gradient[\Point{p}_3]{\|\DNormal{n}\|^2}[\Point{q}]
& = &
\left[
{\frac{
\Normal{n}(\Point{q}_{012})  \perp  \Normal{n}(\Point{q}_{031})
}
{A(\Point{q}_{031})}
}
\times (\Point{q}_1 - \Point{q}_0)
\right]
\end{eqnarray}

\end{plSection}%{Difference in face normals}
%-----------------------------------------------------------------
\begin{plSection}{Inner product between face normals}
\label{sec:normal_dot}

The inner product $\left( \Normal{n}_{012} \bullet \Normal{n}_{031} \right)$
is another important measure of edge curvature.
It is closely related to the squared distance between adjacent normals:
\begin{equation}
\label{eq:normal-distance-dot}
\| \Normal{n}_{012} - \Normal{n}_{031} \|^2
= \| \Normal{n}_{012} \|^2
+ \| \Normal{n}_{031} \|^2
- 2 \left( \Normal{n}_{012} \bullet \Normal{n}_{031} \right)
= 2 \left[ 1 - \left( \Normal{n}_{012} \bullet \Normal{n}_{031} \right) \right]
\end{equation}

The function $f(\Point{p}) = 1 - \left( \Normal{n}_{012} \bullet \Normal{n}_{031} \right)$
achieves its minimum, $0$, on flat face pairs,
and its maximum, $2$, on face pairs that are folded back on themselves.
It's a reasonable choice the total bending or curvature of a surface.
And $\Derivative{f} = - \Derivative{\left( \Normal{n}_{012} \bullet \Normal{n}_{031} \right)}$.

The derivative of
$\left( \Normal{n}_{012} \bullet \Normal{n}_{031} \right)$
can be calculated using equations \cref{eq:dot_derivative} and
\cref{eq:unit_normal_derivative}:
\begin{eqnarray}
\label{normal_dot_derivative}
\Derivative{\left( \Normal{n}_{012} \bullet \Normal{n}_{031} \right)}
[\Point{q}]
& = & \Normal{n}(\Point{q}_{031}) 
\bullet \Derivative{\Normal{n}_{012}}{\Point{q}} 
+ \Normal{n}(\Point{q}_{012}) 
\bullet \Derivative{\Normal{n}_{031}}[\Point{q}]
\\
\nonumber \\
& = &
\Normal{n}(\Point{q}_{031}) \bullet
\frac{
\Identity - 
\left(\Normal{n}(\Point{q}_{012}) \otimes \Normal{n}(\Point{q}_{012}) 
\right)}{\| \Vector{a}(\Point{q}_{012}) \|}
\; \Derivative{\Vector{a}_{012}}[\Point{q}]
\nonumber \\
& + &
\Normal{n}(\Point{q}_{012}) \bullet
\frac{\Identity - \left(\Normal{n}(\Point{q}_{031}) \otimes \Normal{n}(\Point{q}_{031}) \right)}{\| \Vector{a}(\Point{q}_{031}) \|}
\; \Derivative{\Vector{a}_{031}}[\Point{q}]
\nonumber
\end{eqnarray}

As in \cref{sec:normal_difference}, we can write the partial gradients
without reference to an argument:
\begin{eqnarray}
\label{eq:normal_dot_gradient}
\Gradient[\Point{p}_0]{(\Normal{n}_{012} \bullet \Normal{n}_{031})}[\Point{q}]
& = \; &
\frac{ \Normal{n}(\Point{q}_{031}) - 
\left[ \Normal{n}(\Point{q}_{012}) \bullet \Normal{n}(\Point{q}_{031}) \right] 
\Normal{n}(\Point{q}_{012}) }
{\| \Vector{a} (\Point{q}_{012}) \| }
\times (\Point{q}_2 - \Point{q}_1)
\\
& \; + &
\frac{ \Normal{n}(\Point{q}_{012}) - \left[ \Normal{n}(\Point{q}_{012}) \bullet \Normal{n}(\Point{q}_{031}) \right] \Normal{n}(\Point{q}_{031})  }
{\| \Vector{a} (\Point{q}_{031}) \| }
\times (\Point{q}_1 - \Point{q}_3)
\nonumber \\
& & \nonumber \\
\Gradient[\Point{p}_1]{(\Normal{n}_{012} \bullet \Normal{n}_{031})}[\Point{q}]
& = \; &
\frac{ \Normal{n}(\Point{q}_{031}) - \left[ \Normal{n}(\Point{q}_{012}) \bullet \Normal{n}(\Point{q}_{031}) \right] \Normal{n}(\Point{q}_{012})  }
{\| \Vector{a} (\Point{q}_{012}) \| }
\times (\Point{q}_0 - \Point{q}_2)
\nonumber \\
& \; + &
\frac{ \Normal{n}(\Point{q}_{012}) - \left[ \Normal{n}(\Point{q}_{012}) \bullet \Normal{n}(\Point{q}_{031}) \right] \Normal{n}(\Point{q}_{031})   }
{\| \Vector{a} (\Point{q}_{031}) \| }
\times (\Point{q}_3 - \Point{q}_0)
\nonumber \\
& & \nonumber \\
\Gradient[\Point{p}_2]{(\Normal{n}_{012} \bullet \Normal{n}_{031})}[\Point{q}]
& = \; &
\frac{ \Normal{n}(\Point{q}_{031}) - \left[ \Normal{n}(\Point{q}_{012}) \bullet \Normal{n}(\Point{q}_{031}) \right] \Normal{n}(\Point{q}_{012})  }
{\| \Vector{a} (\Point{q}_{012}) \| }
\times (\Point{q}_1 - \Point{q}_0)
\nonumber \\
& & \nonumber \\
\Gradient[\Point{p}_3]{(\Normal{n}_{012} \bullet \Normal{n}_{031})}[\Point{q}]
& = \; &
\frac{ \Normal{n}(\Point{q}_{012}) - \left[ \Normal{n}(\Point{q}_{012}) \bullet \Normal{n}(\Point{q}_{031}) \right] \Normal{n}(\Point{q}_{031}) }
{\| \Vector{a} (\Point{q}_{031}) \| }
\times (\Point{q}_0 - \Point{q}_1)
\nonumber
\end{eqnarray}

This can be simplified using the fact that
\(\Normal{n}_i \perp \Normal{n}_j = \Normal{n}_i - \left[ \Normal{n}_i \bullet \Normal{n}_j \right] \Normal{n}_j\), for unit vectors,
and the face area \(A(\Point{q}) = \frac{1}{2} \| \Vector{a}(\Point{q}) \|\):
\begin{eqnarray}
\label{eq:simplified_normal_dot_gradient}
\Gradient[\Point{p}_0]{(\Normal{n}_{012} \bullet \Normal{n}_{031})}[\Point{q}]
& = \;\;\; &
\frac{\left[ \Normal{n}(\Point{q}_{031}) \perp \Normal{n}(\Point{q}_{012}) \right]}{2A(\Point{q}_{012})}
\times (\Point{q}_2 - \Point{q}_1)
\\
& \;\;\; + &
\frac{\left[ \Normal{n}(\Point{q}_{012}) \perp \Normal{n}(\Point{q}_{031}) \right]}{2A(\Point{q}_{031})}
\times (\Point{q}_1 - \Point{q}_3)
\nonumber \\
& & \nonumber \\
\Gradient[\Point{p}_1]{(\Normal{n}_{012} \bullet \Normal{n}_{031})}[\Point{q}]
& = \;\;\; &
\frac{\left[ \Normal{n}(\Point{q}_{031}) \perp \Normal{n}(\Point{q}_{012}) \right]}{2A(\Point{q}_{012})}
\times (\Point{q}_0 - \Point{q}_2)
\nonumber \\
& \;\;\; + &
\frac{\left[ \Normal{n}(\Point{q}_{012}) \perp \Normal{n}(\Point{q}_{031}) \right]}{2A(\Point{q}_{031})}
\times (\Point{q}_3 - \Point{q}_0)
\nonumber \\
& & \nonumber \\
\Gradient[\Point{p}_2]{(\Normal{n}_{012} \bullet \Normal{n}_{031})}[\Point{q}]
& = \;\;\; &
\frac{\left[ \Normal{n}(\Point{q}_{031}) \perp \Normal{n}(\Point{q}_{012}) \right]}{2A(\Point{q}_{012})}
\times (\Point{q}_1 - \Point{q}_0)
\nonumber \\
& & \nonumber \\
\Gradient[\Point{p}_3]{(\Normal{n}_{012} \bullet \Normal{n}_{031})}[\Point{q}]
& = \;\;\; &
\frac{\left[ \Normal{n}(\Point{q}_{012}) \perp \Normal{n}(\Point{q}_{031}) \right]}{2A(\Point{q}_{031})}
\times (\Point{q}_0 - \Point{q}_1)
\nonumber
\end{eqnarray}

\end{plSection}%{Inner product between face normals}
%-----------------------------------------------------------------
\begin{plSection}{Squared inner product between face normals}
\label{sec:squared_normal_dot}

We can get a more even distribution of bending by giving
a higher weight to sharper edge bends.
A simple way to do that is to square some existing function,
for example: $\left(1 - \Normal{n}_{012} \bullet \Normal{n}_{031}\right)^2$.
The derivative is simply:
\begin{equation}
\Derivative{\left(1 - \Normal{n}_{012} \bullet \Normal{n}_{031}\right)^2}
= -2 \left( 1 - \Normal{n}_{012} \bullet \Normal{n}_{031} \right)
\Derivative{(\Normal{n}_{012} \bullet \Normal{n}_{031})}
\end{equation}

It follows from \cref{eq:simplified_normal_dot_gradient}
that the partial gradients are:
\begin{eqnarray}
\label{eq:squared_normal_dot_gradient}
\Gradient[\Point{p}_0]{\left(1 - \Normal{n}_{012} \bullet \Normal{n}_{031}\right)^2}[\Point{q}]
& = \;\;\; &
\frac{\left( \Normal{n}(\Point{q}_{012}) \bullet \Normal{n}(\Point{q}_{031}) - 1\right)
}
{A(\Point{q}_{012}) }
\left[ \Normal{n}(\Point{q}_{031}) \perp \Normal{n}(\Point{q}_{012}) \right]
\times (\Point{q}_2 - \Point{q}_1)
\\
& \;\;\; + &
\frac{\left( \Normal{n}(\Point{q}_{012}) \bullet \Normal{n}(\Point{q}_{031}) - 1\right)
}{A(\Point{q}_{031})}
\left[ \Normal{n}(\Point{q}_{012}) \perp \Normal{n}(\Point{q}_{031}) \right]
\times (\Point{q}_1 - \Point{q}_3)
\nonumber \\
& & \nonumber \\
\Gradient[\Point{p}_1]{\left(1 - \Normal{n}_{012} \bullet \Normal{n}_{031}\right)^2}[\Point{q}]
& = \;\;\; &
\frac{\left( \Normal{n}(\Point{q}_{012}) \bullet \Normal{n}(\Point{q}_{031}) - 1\right)
}{A(\Point{q}_{012})}
\left[ \Normal{n}(\Point{q}_{031}) \perp \Normal{n}(\Point{q}_{012}) \right]
\times (\Point{q}_0 - \Point{q}_2)
\nonumber \\
& \;\;\; + &
\frac{\left( \Normal{n}(\Point{q}_{012}) \bullet \Normal{n}(\Point{q}_{031}) - 1\right)
}{A(\Point{q}_{031})}
\left[ \Normal{n}(\Point{q}_{012}) \perp \Normal{n}(\Point{q}_{031}) \right]
\times (\Point{q}_3 - \Point{q}_0)
\nonumber \\
& & \nonumber \\
\Gradient[\Point{p}_2]{\left(1 - \Normal{n}_{012} \bullet \Normal{n}_{031}\right)^2}[\Point{q}]
& = \;\;\; &
\frac{\left( \Normal{n}(\Point{q}_{012}) \bullet \Normal{n}(\Point{q}_{031}) - 1\right)
}{A(\Point{q}_{012})}
\left[ \Normal{n}(\Point{q}_{031}) \perp \Normal{n}(\Point{q}_{012}) \right]
\times (\Point{q}_1 - \Point{q}_0)
\nonumber \\
& & \nonumber \\
\Gradient[\Point{p}_3]{\left(1 - \Normal{n}_{012} \bullet \Normal{n}_{031}\right)^2}[\Point{q}]
& = \;\;\; &
\frac{\left( \Normal{n}(\Point{q}_{012}) \bullet \Normal{n}(\Point{q}_{031}) - 1\right)
}{A(\Point{q}_{031})}
\left[ \Normal{n}(\Point{q}_{012}) \perp \Normal{n}(\Point{q}_{031}) \right]
\times (\Point{q}_0 - \Point{q}_1)
\nonumber
\end{eqnarray}

\end{plSection}%{Squared inner product between face normals}
%-----------------------------------------------------------------

\begin{plSection}{Dihedral angle}
\label{sec:Dihedral-angle}

The {\it dihedral angle},
$\theta_d$,
 of an edge is the amount
needed to rotate one face
through the inside of the surface
(in the sense of the orientation of the faces)
onto the other face.

With this definition, the dihedral angle is always
positive, in fact, between $0$ and $2\pi$.
An acutely concave edge has a dihedral angle
of a little less than $2\pi$;
a concave edge where the faces meet in a right angle
$\frac{3\pi}{2}$;
a flat edge is $\pi$;
a convex right angle $\frac{\pi}{2}$;
and an acutely convex edge has a dihedral angle
slightly more than $0$.

The dihedral angle is a simple function of the
signed normal angle, $\theta_n$
(\cref{sec:signed_normal_angle}),
and it is generally more convenient to work
with the signed normal angle directly,
so I will not consider the dihedral angle further.

\end{plSection}%{Dihedral angle}
%-----------------------------------------------------------------

\begin{plSection}{Unsigned normal angle}
\label{sec:unsigned_normal_angle}

The inner product 
$\left( \Normal{n}_{012} \bullet \Normal{n}_{031} \right)$
is the cosine of $\theta_u$,
the (unsigned) angle between the 2 face normals.
Thus the value of the angle is:
\begin{equation}
\theta_u(\Point{p}_0,\Point{p}_1,\Point{p}_2,\Point{p}_3)
= \cos^{-1} \left( \Normal{n}_{012} \bullet \Normal{n}_{031} \right)
\end{equation}

Recall the derivative of the $\cos^{-1}$ is:
\begin{equation}
\frac{d}{\mathit dx} \cos^{-1}(x) = \frac{-1}{\sqrt{1-x^2}}
\end{equation}

The derivative of $\theta_u$ can be calculated using the chain rule
(\cref{eq:chain-rule}) and \cref{eq:normal_dot_gradient}.
\begin{equation}
\Derivative{\theta_u}[\Point{q}]
 = \frac{-1}
{
\sqrt{
1-
\left(
\Normal{n}(\Point{q}_{012}) \bullet \Normal{n}(\Point{q}_{031}) 
\right)^2
} 
}
\; \Derivative{\left( \Normal{n}_{012} \bullet \Normal{n}_{031} \right)}[\Point{q}]
\end{equation}

As above, we can write the partial gradients without reference to an argument:
\begin{equation}
\Gradient[\Point{p}_i]{\theta_u}[\Point{q}]
=
\frac{-1}
{
\sqrt{
1 - 
\left( 
\Normal{n}(\Point{q}_{012}) \bullet \Normal{n}(\Point{q}_{031}) 
\right)^2
} 
}
\; \Gradient[\Point{p}_i]{\left(\Normal{n}_{012}\bullet\Normal{n}_{031}\right)}[\Point{q}]
\end{equation}

\end{plSection}%{Unsigned normal angle}
%-----------------------------------------------------------------

\begin{plSection}{Signed normal angle}
\label{sec:signed_normal_angle}

The unsigned normal angle
($\theta_u$, \cref{sec:unsigned_normal_angle})
doesn't distinguish concave and convex changes in the normal vector,
so it can't be a 1-1 function of the dihedral angle.
More importantly, it's inadequate if we want to construct
a penalty that treats concave and convex edges differently.

It is common to want to make a surface as flat as possible,
which means penalizing normal angles different from zero.
With such 2rd order (curvature minimizing) penalties, 
it may not necessary to distinguish
convex and concave bending.

However, there's considerable evidence that
$3$rd order penalties ---minimizing the {\it variation}
in surface bending, rather than the bending itself---
gives better results in many situations.
To measure bending variation, it's necessary
to use correctly signed angles.

Another reason to use signed angles is to be able
to encourage the surface to bend in a certain
predetermined way at certain edges.
This is discussed in \cref{sec:Bent-edge-neighborhoods}
below.

The {\it signed normal angle,} $\theta_n$,
is essentially just the
unsigned normal angle, multiplied by $-1$ if the surface is
concave at the edge.
To make this a bit more formal,
we can define the signed normal angle to be the amount
of righthanded rotation about the edge tangent 
($\Point{p}_1 - \Point{p}_0$)
needed to bring the right normal ($\Normal{n}_{031}$) to
the left normal ($\Normal{n}_{012}$).
We cut our angle measurement at $\pi \sim -\pi$.

With this definition, the signed normal angle
is between $-\pi$ and $\pi$.
An acutely concave edge has a signed normal angle
of a little more than $-\pi$;
a concave edge where the faces meet in a right angle
$-\frac{\pi}{2}$;
a flat edge is $0$;
a convex right angle $\frac{\pi}{2}$;
and an acutely convex edge has a signed normal angle
slightly less than $\pi$.

The dihedral angle can be expressed simply in terms of the
signed normal angle: $\theta_d = \pi - \theta_n$.

To compute $\theta_n$,
it's easiest to use the informal definition:
\begin{equation}
\theta_n(\Point{p}_0,\Point{p}_1,\Point{p}_2,\Point{p}_3)
= \kappa(\Point{p}_0,\Point{p}_1,\Point{p}_2,\Point{p}_3) \ast \theta_u(\Point{p}_0,\Point{p}_1,\Point{p}_2,\Point{p}_3),
\end{equation}
where $\kappa(\Point{p}_0,\Point{p}_1,\Point{p}_2,\Point{p}_3)$ is
$+1$ if the edge is convex
and
$-1$ if the edge is concave.
One way to determine convexity/concavity
is to look at
$\left(\Point{p}_3 - \Point{p}_2 \right) \bullet \Normal{n}_{031}$
which is positive if the edge is convex
and negative if it is concave.

The derivative of $\theta_n$ are simply $\kappa$
times the derivative of $\theta_u$:
$\Derivative{\theta_n}{\Point{q}} 
= 
\kappa(\Point{q}) \ast \Derivative{\theta_u}[\Point{q}]$.

\end{plSection}%{Signed normal angle}
%-----------------------------------------------------------------

\begin{plSection}{Signed angle and normal distance}
\label{sec:Signed-angle-and-normal-distance}

We can tie the signed normal angle back 
to the distance between normal vectors,
and save the time and complexity
of computing the $\cos^{-1}$ and its derivatives,
by considering a simple transform.

First, note that $\sin(\frac{\theta_n}{2})$
is a monotone function of $\theta_n$,
ranging from $-1$, when $\theta_n = -\pi$,
to $1$ when $\theta_n = \pi$.
Also note that 
\begin{equation}
\sin(\frac{\theta_n}{2}) 
 = 
\sign(\theta_n) 
\left[ 
\frac{1 - \cos(\theta_n)}{2}
\right]^{\frac{1}{2}}
 = 
\sign(\theta_n) 
\left[ 
\frac{1-\left(\Normal{n}_{012}\bullet\Normal{n}_{031}\right)}{2} 
\right]^{\frac{1}{2}},
\end{equation}
which means that
\begin{equation}
\sin^2(\frac{\theta_n}{2}) 
= 
\frac{1-\left(\Normal{n}_{012}\bullet\Normal{n}_{031}\right)}{2},
\end{equation}
which is proportional to the distance between the adjacent normals
(see \cref{eq:normal-distance-dot}).

Note that the function that promotes evenly 
distributed bending described in \cref{sec:squared_normal_dot},
is proportional to $\sin^4(\frac{\theta_n}{2})$. 

\end{plSection}%{Signed angle and normal distance}
%-----------------------------------------------------------------

\begin{plSection}{Bent edge neighborhoods}
\label{sec:Bent-edge-neighborhoods}

In many problems \cite{HoppeEtal:1994:SIGGRAPH,Hoppe:1994:Phd},
it's important to be able to allow the surface to have
sharp creases and cusps.
A simple way to achieve this is to mark
certain edges as ``sharp'',
and then not evaluate any bending penalty
on those edges.
With this approach, there are then 2 kinds of edges:
sharp edges, where any bend angle is equally acceptable,
and smooth edges, 
where any deviation from zero bending is penalized.

However, it's not often the case that all angles
are equally desirable for a ``sharp'' edge,
or that no angle, other than $0$, is acceptable
for a ``smooth'' edge.
It's often true that we know that a given crease
must be, for example, convex ($\theta_n > 0$)
but with bounded acuteness ($\theta_n < \alpha < \pi$).
Or we may know that a given edge should be close to
a convex right angle 
($\left( \frac{\pi}{2} - \delta \right) 
< \theta_n < 
\left( \frac{\pi}{2} + \delta \right)$)
for some small positive angle $\delta$.

Enforcing hard constraints on $\theta_n$
is relatively difficult,
but it is easy to convert any function
of the signed normal angle, $f(\theta_n)$,
that penalizes non-zero normal angles,
into one that penalizes angles 
outside an interval, $\left[\theta_0,\theta_1\right]$,
of preferred angles.
($-\pi \leq \theta_0 \leq \theta_1 \leq \pi$.)

For example,
consider the function of \cref{sec:squared_normal_dot},
which is proportional to $\sin^4(\frac{\theta_n}{2})$. 
Let $s = \sin(\frac{\theta_n}{2})$,
a monotone function of $\theta_n$ 
that ranges over $\left[ -1, 1 \right]$.

Then, 
given an interval 
$\left[ \theta_0, \theta_1 \right] 
\subseteq 
\left[ -\pi, \pi \right]$,
let the {\it bent} function be:
\begin{equation}
f(s)
= 
\left\{
\begin{array}{cr}
\left[ {  
\frac{\textstyle
s^2 - s_0^2}
{\textstyle
1 - s_0^2}
  } \right]^2 
& -\pi \leq \theta_n \leq \theta_0
\\
\\
0                            
& \theta_0 \leq \theta_n \leq \theta_1
\\
\\
\left[
\frac{\textstyle s^2 - s_1^2}{\textstyle 1 - s_1^2}
\right]^2  
& \theta_1 \leq \theta_n \leq \pi
\end{array}
\right.
\end{equation}
The reason for stretching $s^2$ rather than $s$
is that we get a relatively simple expression
in $\Normal{n}_{012} \bullet \Normal{n}_{031}$:
\begin{equation}
f(s)
= 
\left\{
\begin{array}{cr}
\left[
\frac{\textstyle c_0 - \left(\Normal{n}_{012} \bullet \Normal{n}_{031}\right) } 
{\textstyle c_0 + 1}
\right]^2
& -\pi \leq \theta_n \leq \theta_0
\\
\\
0                            
& \theta_0 \leq \theta_n \leq \theta_1
\\
\\
\left[ 
\frac{
\textstyle{
c_1 - \left(\Normal{n}_{012} \bullet \Normal{n}_{031}\right)
}
} 
{\textstyle{ c_1 + 1}} 
\right]^2
& \theta_1 \leq \theta_n \leq \pi
\end{array}
\right.
\end{equation}
where $c_0 = \cos(\theta_0)$
and $c_1 = \cos(\theta_1)$.
Recall that 
$\cos(\theta_n) = 
\left(\Normal{n}_{012} \bullet \Normal{n}_{031}\right)$.

\end{plSection}%{Bent edge neighborhoods}
%-----------------------------------------------------------------

\begin{plSection}{Weighting bend measures by length}
\label{sec:length_weighted_bend}

Let $f(\Point{p}) = 
f(\Point{p}_0,\Point{p}_1,\Point{p}_2,\Point{p}_3)$ 
be some measure of the
bend across the edge connecting $(\Point{p}_0,\Point{p}_1)$.
It may be useful to weight the bend measure by the edge length:
$\| \Point{p}_0 - \Point{p}_1 \| \ast f(\Point{p}_0,\Point{p}_1,\Point{p}_2,\Point{p}_3)$.
The partial gradients of the weighted measure are:
\begin{eqnarray}
\label{eq:length_weighted_bend_gradient}
\Gradient[\Point{p}_0]{\left( \|\Point{p}_0 - \Point{p}_1 \| f(\Point{p}) \right)}[\Point{q}]
& = &
\frac{f(\Point{q})}{\|\Point{q}_0 - \Point{q}_1\|} \left( \Point{q}_0 - \Point{q}_1 \right)
+ \| \Point{q}_0 - \Point{q}_1 \| \Gradient[\Point{p}_0]{f}[\Point{q}]
\\
& & \nonumber \\
\Gradient[\Point{p}_1]{\left( \|\Point{p}_0 - \Point{p}_1 \| f(\Point{p}) \right)}[\Point{q}]
& = &
\frac{f(\Point{q})}{\|\Point{q}_0 - \Point{q}_1\|} \left( \Point{q}_1 - \Point{q}_0 \right)
+ \| \Point{q}_0 - \Point{q}_1 \| \Gradient[\Point{p}_1]{f}[q]
\nonumber \\
& & \nonumber \\
\Gradient[\Point{p}_2]{\left( \|\Point{p}_0 - \Point{p}_1 \| f(\Point{p}) \right)}[\Point{q}]
& = &
\| \Point{q}_0 - \Point{q}_1 \| \Gradient[\Point{p}_2]{f}[q]
\nonumber \\
& & \nonumber \\
\Gradient[\Point{p}_3]{\left( \|\Point{p}_0 - \Point{p}_1 \| f(\Point{p}) \right)}[\Point{q}]
& = &
\| \Point{q}_0 - \Point{q}_1 \| \Gradient[\Point{p}_3]{f}[\Point{q}]
\nonumber
\end{eqnarray}


\end{plSection}%{Weighting bend measures by length}
%-----------------------------------------------------------------

\begin{plSection}{Rate of bending}
\label{sec:Rate-of-bending}

It may be useful to weight edge bends should
by the length of the edge ---
a long sharp edge should cost more than a short sharp edge.
It may also be useful to weight inversely by the
distance to nearby edges --- two edges close together
means more curvature than the same two edges farther apart.
A measure has both these properties, at least approximately,
is:
\begin{equation}
{\mathrm dN2L2A (\Point{p}) }
=
\| \Normal{n}(\Point{p}_{012}) - \Normal{n}(\Point{p}_{031}) \|^2
\| \Point{p}_0 - \Point{p}_1 \|^2
\left[
\frac{1}{A(\Point{p}_{012})} +
\frac{1}{A(\Point{p}_{031})}
\right]
\end{equation}
where $\| \Normal{n}(\Point{p}_{012}) - \Normal{n}(\Point{p}_{031}) \|^2$ is the squared distance between
adjacent normals, as in \cref{eq:deltan}.
(Note the similarity of the weighting to the aspect ratio
measure discussed in \cref{sec:Squared-edge-lengths-over-area}.)

\begin{eqnarray}
\Gradient[\Point{p}]{dN2L2A}[\Point{q}]
& = &
\Gradient[\Point{p}]{\left(\| \Normal{n}(\Point{p}_{012}) - \Normal{n}(\Point{p}_{031}) \|^2 \right)}[\Point{q}]
\| \Point{q}_0 - \Point{q}_1 \|^2
\left[
\frac{1}{A(\Point{q}_{012})} +
\frac{1}{A(\Point{q}_{031})}
\right]
\\
& + &
\| \Normal{n}(\Point{q}_{012}) - \Normal{n}(\Point{q}_{031}) \|^2
\Gradient[\Point{p}]{\left(\| \Point{p}_0 - \Point{p}_1 \|^2\right)}[\Point{q}]
\left[
\frac{1}{A(\Point{q}_{012})} +
\frac{1}{A(\Point{q}_{031})}
\right]
\nonumber
\\
& + &
\| \Normal{n}(\Point{q}_{012}) - \Normal{n}(\Point{q}_{012}) \|^2
\| \Point{q}_0 - \Point{q}_1 \|^2
\left[
\Gradient[\Point{p}]{\left( \frac{1}{A(\Point{p}_{012})} \right)}[\Point{q}] +
\Gradient[\Point{p}]{\left( \frac{1}{A(\Point{p}_{031})} \right)}[\Point{q}]
\right]
\nonumber
\end{eqnarray}

The partial gradients,
$\Gradient[\Point{p}_i]{\left(\| \Normal{n}(\Point{p}_{012}) - \Normal{n}(\Point{p}_{031}) \|^2 \right)}[\Point{q}]$,
is given in \cref{eq:normal-difference-gradient}.

$\Gradient[\Point{p}_0]
{\left(\| \Point{p}_0 - \Point{p}_1 \|^2\right)}
[\Point{q}]
 = 2 \left( \Point{q}_0 - \Point{q}_1 \right)$;
$\Gradient[\Point{p}_1]
{\left(\| \Point{p}_0 - \Point{p}_1 \|^2\right)
}[\Point{q}] 
= 2 \left( \Point{q}_1 - \Point{q}_0 \right)$;
and the other 2 partial gradients are zero.

Using \cref{eq:area_partial_derivative}
and the chain rule, we have:
\begin{eqnarray}
\label{eq:inverse-area-gradient-012}
\Gradient[\Point{p}_0]
{\left( \frac{1}{A(\Point{p}_{012})} \right)}
[\Point{q}]
& = &
\frac{
\Normal{n}(\Point{q}_{012}) \times 
\left( \Point{q}_1 - \Point{q}_2 \right)
}
{2 A(\Point{q}_{012})^2 }
\\
\Gradient[\Point{p}_1]
{\left( \frac{1}{A(\Point{p}_{012})} \right)}
[\Point{q}]
& = &
\frac{
\Normal{n}(\Point{q}_{012}) \times
 \left( \Point{q}_2 - \Point{q}_0 \right)
 }
{2 A(\Point{q}_{012})^2 } 
\nonumber
\\
\Gradient[\Point{p}_2]
{\left( \frac{1}{A(\Point{p}_{012})} \right)}
[\Point{q}]
& = &
\frac{
\Normal{n}(\Point{q}_{012}) \times 
\left( \Point{q}_0 - \Point{q}_1 \right)
}
{2 A(\Point{q}_{012})^2 } 
\nonumber
\end{eqnarray}
\begin{eqnarray}
\label{eq:inverse-area-gradient-031}
\Gradient[\Point{p}_0]
{\left( \frac{1}{A(\Point{p}_{031})} \right)}
[\Point{q}]
& = &
\frac{
\Normal{n}(\Point{q}_{031}) \times
 \left( \Point{q}_3 - \Point{q}_1 \right)
 }
{2 A(\Point{q}_{031})^2 }
\\
\Gradient[\Point{p}_1]
{\left( \frac{1}{A(\Point{p}_{031})} \right)}
[\Point{q}]
& = &
\frac{
\Normal{n}(\Point{q}_{031}) \times 
\left( \Point{q}_0 - \Point{q}_3 \right)
}
{2 A(\Point{q}_{031})^2 }
\nonumber
\\
\Gradient[\Point{p}_3]
{\left( \frac{1}{A(\Point{p}_{031})} \right)}
[\Point{q}]
& = &
\frac{
\Normal{n}(\Point{q}_{031}) \times 
\left( \Point{q}_1 - \Point{q}_0 \right)
}
{2 A(\Point{q}_{031})^2 }
\nonumber
\end{eqnarray}

\end{plSection}%{Rate of bending}
%-----------------------------------------------------------------
\end{plSection}%{Functions of the edges}
