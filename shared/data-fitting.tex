\begin{plSection}{Data Fitting}
\label{sec:data-fitting}

We can fit a triangular mesh, $\Set{M}$, to a set of data points, 
$\{\Vector{d}_i; i=0 \ldots n-1\}$
by minimizing the sum of the $l_2$ distances from the points to the mesh:
\begin{equation}
f(\Set{M}) = \sum_{i=0}^{n-1} 
\|\Vector{d}_i-\Projection_\Set{M} (\Vector{d}_i)\|^2 ,
\end{equation}
where $\Projection_\Set{M} (\Vector{d}_i)$ 
is the point on $\Set{M}$ closest to $\Vector{d}_i$,
that is, the {\em projection} of $\Vector{d}_i$ on $\Set{M}$.
Note that $f:\Reals^{3n} \mapsto \Re$,
where $n$ is the number of vertices in $\Set{M}$.

We compute $\Projection_\Set{M} (\Vector{d})$ by minimizing  
$\|\Vector{d}-\Projection_\Simplex{s} (\Vector{d})\|^2$
over all simplices $\Simplex{s} \in \Set{M}$.
We need only consider the faces of $\Set{M}$,
those edges not in any face,
and the vertices not in any edge,
because the closest point on a face must at least
as the closest point on any of its edges,
and similarly for vertices.
More generally, spatial binning of the simplices and data can greatly
reduce the number of simplices that need to be examined.

The following sections consider the projection of a single
data point $\Vector{d}$ on a mesh $\Set{M}$,
and the gradient of the squared distance,
as a function of the vertex positions $\Vector{p}(\Simplex{v})$.

Unfortunately, derivatives of the distance function are not continuous.
Second derivative discontinuities occur
when a data point is on the boundary
of a 'watershed' region, the set of points
projecting on a vertex or the interior of an edge or face.
Gradient discontinuities are encountered
when a data point is equidistant from 2 distinct closest mesh points.

%-----------------------------------------------------------------
\begin{plSection}{Distance to vertex}
\label{sec:Distance-to-vertex}

Let $\Vector{p} = \Vector{p}(\Simplex{v})$ 
be the position of a particular vertex $\Simplex{v}$,
and $\Vector{d}$ the 3d data point.
It follows from \cref{eq:l2-gradient} that
\begin{equation}
\label{eq:vertex-distance-gradient}
\Gradient[\Vector{p}]{\|\Vector{p}-\Vector{d}\|^2}[\Vector{q}] 
= 2 ( \Vector{q}-\Vector{d} ).
\end{equation}

The distance to the nearest vertex in a set of vertices $\Set{V}$ is:
$\min_{\Simplex{v} \in \Set{V}} 
\|\Vector{p}(\Simplex{v})-\Vector{d}\|^2$.
If $\Simplex{v}^{\mathrm min}$ is the minimizing vertex,
and
$\Vector{p}^{\mathrm min}$ its position,
then the partial gradient with respect
to the position of any other vertex is zero,
and the partial gradient with respect to $\Vector{p}^{\mathrm min}$
is given in \cref{eq:vertex-distance-gradient}.
Note that the gradient is only defined and continuous
while $\Vector{d}$ is within the interior of the
Voronoi regions surrounding the vertices.

\end{plSection}%{Distance to vertex}
%-----------------------------------------------------------------
\begin{plSection}{Distance to edge}
\label{sec:Distance-to-edge}

Let the edge $\Simplex{e}$ have end points 
$\Vector{p} = (\Vector{p}_0, \Vector{p}_1) \in \Reals^6$.
We can write the projection of a data point 
$\Vector{d}$ on $\Simplex{e}$ as:
\begin{equation}
\Projection_\Vector{p} (\Vector{d}) 
= b_0(\Vector{p}) \Vector{p}_0 + b_1(\Vector{p}) \Vector{p}_1
\end{equation}
where
\begin{eqnarray}
b_0(\Vector{p}) & = &
\min\left(0,\max\left(1,
\frac{ (\Vector{d}-\Vector{p}_1) \bullet (\Vector{p}_0-\Vector{p}_1) }
{\|\Vector{p}_0-\Vector{p}_1\|^2 }
\right) \right) \\
b_1(\Vector{p}) & = & 1-b_0(\Vector{p})
\nonumber
\end{eqnarray}

\begin{eqnarray}
\label{eq:edge-distance-gradient-derivation}
\Derivative[\Vector{p}_0]
{\|\Projection_{\Vector{p}} (\Vector{d})-\Vector{d}\|^2 }
[\Vector{q}]
& = &
2 \left( 
\Projection_{\Vector{q}} (\Vector{d})-\Vector{d} 
\right)^\dagger
\Derivative[\Vector{p}_0]
{\Projection_{\Vector{p}} (\Vector{d}) }
[\Vector{q}]
\\
& = &
2 \left( 
\Projection_{\Vector{q}} (\Vector{d})-\Vector{d}
 \right)^\dagger
\Derivative[\Vector{p}_0]
{\left[ 
b_0(\Vector{p})\Vector{p}_0 + b_1(\Vector{p})\Vector{p}_1 
\right]}
[\Vector{q}]
\nonumber \\
& = &
2 \left( 
\Projection_{\Vector{q}} (\Vector{d})-\Vector{d}
 \right)^\dagger
\Derivative[\Vector{p}_0]
{\left[ 
b_0(\Vector{p})\Vector{p}_0 + (1-b_0(\Vector{p}))\Vector{p}_1 
\right]}
[\Vector{q}]
\nonumber \\
& = &
2 \left( \Projection_{\Vector{q}} (\Vector{d})-\Vector{d}
 \right)^\dagger
\Derivative[\Vector{p}_0]
{\left[ b_0(\Vector{p})(\Vector{p}_0-\Vector{p}_1) \right]}
[\Vector{q}]
\nonumber \\
& = &
2 \left( 
\Projection_{\Vector{q}} (\Vector{d})-\Vector{d} \right)^\dagger
\left[ b_0(\Vector{q}) \Identity + (\Vector{q}_0-\Vector{q}_1)
 \otimes 
 \Gradient[\Vector{p}_0]{b_0(\Vector{p})}[\Vector{q}] \right]
\nonumber
\end{eqnarray}
Because
 $\left( \Projection_{\Vector{q}}
  (\Vector{d})-\Vector{d} \right)$ is orthogonal to
$\left( \Vector{q}_0-\Vector{q}_1 \right)$, we get:
\begin{eqnarray}
\label{eq:edge-distance-gradient}
\Gradient[\Vector{p}_0]
{\|\Projection_{\Vector{p}} (\Vector{d})-\Vector{d}\|^2}
[\Vector{q}]
& = & 2 b_0(\Vector{q}) 
\left[ \Projection_{\Vector{q}} (\Vector{d})-\Vector{d} \right]
\\
\Gradient[\Vector{p}_1]
{\|\Projection_{\Vector{p}} (\Vector{d})-\Vector{d}|^2 }
[\Vector{q}]
& = & 2 b_1(\Vector{q}) 
\left[ \Projection_\Vector{q} (\Vector{d})-\Vector{d} \right]
\nonumber
\end{eqnarray}

As in the vertex case,
the distance to the nearest edge in a set of edges $\Set{E}$ is:
\begin{equation}
\|\Projection_{\Set{E}} (\Vector{d})-\Vector{d}|^2
 = \min_{\Simplex{e} \in \Set{E}}
 \|\Projection_{\Vector{p}(\Simplex{e})}
  (\Vector{d})-\Vector{d}\|^2
\end{equation}
If $\Simplex{e}^{\min}$ is the minimizing edge,
$\Simplex{v}_0^{\min}$ and $\Simplex{v}_1^{\min}$ its vertices,
and $\Vector{p}_0^{\min}$ and $\Vector{p}_1^{\min}$
the corresponding endpoints,
then the partial gradient with respect to
the position of any
other vertex is zero,
and the partial gradient with respect 
to $\Vector{p}_0^{\min}$ and $\Vector{p}_1^{\min}$
is given in \cref{eq:edge-distance-gradient}.

The total gradient is defined and continuous
when $\Vector{d}$ is within the union of the watershed regions
of $\Simplex{e}^{\min}$ and its vertices.
It is also continuous where the watershed of one of the vertices
meets the watershed of any of the edges containing that vertex.
It is not if $\Vector{d}$ lies on the boundary of the
watershed of $\Simplex{e}^{\min}$ and the watershed of an
edge with which it does not share a vertex.

\end{plSection}%{Distance to edge}
%-----------------------------------------------------------------
\begin{plSection}{Distance to face}
\label{sec:Distance-to-face}

Let the face $\Simplex{f}$ have corner points 
$\Vector{p} = (\Vector{p}_0, \Vector{p}_1, \Vector{p}_2) \in \Reals^9$.
As in the edge case,
we can write the projection of a data point $\Vector{d}$ on $\Simplex{f}$
in terms of the barycentric coordinates as:
\begin{equation}
\Projection_\Vector{p} (\Vector{d}) = 
b_0(\Vector{p}) \Vector{p}_0 + b_1(\Vector{p}) \Vector{p}_1 
+ b_2(\Vector{p}) \Vector{p}_2,
\end{equation}
and, by an argument similar to that used in
\cref{eq:edge-distance-gradient-derivation},
we can show that
\begin{eqnarray}
\label{eq:face-distance-gradient}
\Gradient[\Vector{p}_0]
{\|\Projection_{\Vector{p}} (\Vector{d})-\Vector{d}\|^2 }
[\Vector{q}]
& = & 2 b_0(\Vector{q}) 
\left[ \Projection_{\Vector{q}} (\Vector{d})-\Vector{d} \right]
\\
\Gradient[\Vector{p}_1]
{\|\Projection_{\Vector{p}} (\Vector{d})-\Vector{d}|^2 }
[\Vector{q}]
& = & 2 b_1(\Vector{q}) 
\left[ \Projection_\Vector{q} (\Vector{d})-\Vector{d} \right]
\nonumber
\\
\Gradient[\Vector{p}_2]
{\|\Projection_{\Vector{p}} (\Vector{d})-\Vector{d}|^2 }
[\Vector{q}]
& = & 2 b_2(\Vector{q}) 
\left[ \Projection_\Vector{q} (\Vector{d})-\Vector{d} \right]
\nonumber
\end{eqnarray}

Computing the barycentric coordinates for the projection
on a face (triangle) is slightly more complicated than
for an edge (line segment).

First center the problem by letting
$\Vector{v} = \Vector{p}-\Vector{p}_0$,
$\Vector{v}_1 = \Vector{p}_1-\Vector{p}_0$, 
and $\Vector{v}_2 = \Vector{p}_2-\Vector{p}_0$.
Then compute the raw, unbounded barycentric coordinates
of the projection of $\Vector{p}$ onto the plane
spanned by the triangle:
\begin{eqnarray}
r_0(\Vector{p}) & = & 1-r_1(\Vector{p})-r_2(\Vector{p})
\\
r_1(\Vector{p}) & = & v \bullet 
\frac{\Vector{v}_1 \perp \Vector{v}_2}
{\|\Vector{v}_1 \perp \Vector{v}_2\|^2}
\nonumber
\\
r_2(\Vector{p}) & = & v \bullet 
\frac{\Vector{v}\Vector{v} \perp \Vector{v}_1} 
{\|\Vector{v}_2 \perp \Vector{v}_1\|^2}
\nonumber
\end{eqnarray}
To correctly bound the raw coordinates to numbers between 0 and 1,
we need to determine whether the projected point is in
the interior of the triangle, on one of the edges,
or on one of the vertices.

\begin{description}

\item[Vertex case:]
If any 2 of the $r_i$ are negative,
then $\Vector{p}$ projects on the remaining vertex.
Set the 2 $b_i$ corresponding to the negative $r_i$
to 0 and the remaining $b_i$ to 1.

\item[Edge case:]
If any single 1 of the $r_i$ is negative,
then $\Vector{p}$ projects on the opposite edge.
Set the $b_i$ corresponding to the negative $r_i$
to 0.
Go to \cref{sec:Distance-to-edge} to see
how to compute the remaining barycentric coordinates
by projecting on the edge

\item[Interior case:]
If none of the $r_i$ is negative,
then $\Vector{p}$ projects on the interior
and each $b_i = r_i$

\end{description}
\end{plSection}%{Distance to face}
%-----------------------------------------------------------------
\end{plSection}%{Data Fitting}
