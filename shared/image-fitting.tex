\begin{plSection}{Image Fitting}
\label{sec:image-fitting}

In this section, I discuss fitting a polygon to a binary classification
image, or a monochrome 'probability' image.
In the first case, the pixel values are either $0$ or $1$,
indicating assignment to $1$ of $2$ possible classes.
In the monochrome case, I take the the value ($0--255$) to indicate
the probability that the pixel belong to class $1$,
or some other score, such as the fraction of the pixel area that belongs
to class $1$.

The basic idea is to optimize the polygon's vertex positions
to maximize the agreement of the inside-outside pixel classification
induced by the polygon with the classification, or class score
given by the image.

\begin{plSection}{Inside-Outside}
\label{sec:inside-outside}

I use a non-standard definition of polygon inside-outside,
based on projection:

\begin{plDiagram}
{Convex vertex}
{ConvexVertex}
\centering
\begin{verbatim}
                            .p
                           .
                   e01    .
             v0 o--------o
                       v1 \
                           \ e12
                            \
                             o v2
\end{verbatim}
\end{plDiagram}

\begin{plDiagram}
{Concave vertex}
{ConcaveVertex}
\centering
\begin{verbatim}
                             o v2
                            /
                           / e12
                          /
             v0 o--------o v1
                   e01    .
                           .
                            .p
\end{verbatim}
\end{plDiagram}

Suppose we have a polygon, $\Set{P}$, consisting of a set of vertices,
$\{\Simplex{v}_i\}$, and consistently directed edges $\{\Simplex{e}_{i,i+1}:(\Simplex{v}_i,\Simplex{v}_{i+1})\}$.
To determine whether a point $\Vector{p}$ is inside or outside,
find the projection of (the closest point to) $\Vector{p}$ on the polygon.
If the projection is on a convex vertex (see \ref{diagram:ConvexVertex}),
then the point is outside.
If the projection is on a concave vertex (see \ref{diagram:ConcaveVertex}),
then the point is inside.
If the projection is to the interior of edge $\Simplex{e}_{i,i+1}$,
then the point is outside if the cross product
$(\Simplex{v}_i-\Vector{p}) \times (\Simplex{v}_{i+1}-\Vector{p}) > 0$.

The {\em signed distance,} $d(\Vector{p}_i,\Set{P})$, from $\Vector{p}$ to the polygon $\Set{P}$,
is the distance to the closest point, times $-1$ if $\Vector{p}$ is inside $\Set{P}$.
I choose this sign so that the positive part of the signed distance
is the same as the distance to the set corresponding to the polygon interior.
Correspondingly, I define a classified pixel's sign $\sign(\Vector{p})$ to be +1
if the pixel is in class 0 and -1 if the pixel is in class 1,
because it's most common to consider the interior of the shape
to be class 1.

\end{plSection}%{Inside-Outside}
%-----------------------------------------------------------------
\begin{plSection}{Counting penalties}
\label{sec:counting-penalties}

\end{plSection}%{Counting penalties}
%-----------------------------------------------------------------
\begin{plSection}{Signed distance penalties}
\label{sec:signed-distance-penalties}

In this section, I describe a class of penalties the depend on the
signed distance of a classified pixel to the polygon.

These penalties sum over (a sample of) the pixels ($\Vector{p}_i$) in the class image:
\begin{equation}
f(\Set{P}) = \sum_{\Vector{p}_i} \phi( \sign(\Vector{p}_i) \ast d(\Vector{p}_i,\Set{P}) ) ,
\end{equation}

The function $\phi(x)$ is typically zero for all $x \le 0$,
monotone increasing in $x$, and bounded by $1$ as $x\rightarrow\infty$.
$\frac{d\phi}{dx}\mid_{x=0} = 0$ and $\frac{d\phi}{dx}\mid_{x} \rightarrow 0$
as $x\rightarrow\infty$.

Examples are:

\begin{eqnarray}
\phi_s(x) & = 3 \left(\frac{x}{r}\right)^2 - 2 \left(\frac{x}{r}\right)^3 & {x \geq 0} \\
          & = 0 & {x \leq 0}
\end{eqnarray}

\begin{eqnarray}
\phi_g(x) & = 1 - e^{ \frac{-x^2}{2\pi r^2} } & x \geq 0 \\
          & = 0                               & x \leq 0 \nonumber
\end{eqnarray}

\end{plSection}%{Signed distance penalties}
%-----------------------------------------------------------------
\end{plSection}%{Image Fitting}
